\pagebreak
\section{Đánh giá và phân tích kết quả}

\subsection{Kết quả dự đoán của mô hình với từng loại feature}

\begin{figure}[H]
\centering
\includegraphics[width=\linewidth]{img/comparison_all_metrics.png}
\caption{So sánh tổng quan Accuracy, Precision, Recall và F1-score giữa các feature}
\label{fig:compare_all_metrics}
\end{figure}

Đầu tiên, ta có thể thấy rằng cả ba dạng feature (PIXEL, BLOCK\_AVG, EDGE) đều cho kết quả dự đoán khá tốt, với độ chính xác dao động trong khoảng 0.87 - 0.92. Tuy nhiên, sự khác biệt giữa chúng là rõ ràng:

\begin{itemize}
    \item \textbf{PIXEL} đạt Accuracy cao nhất $\approx 0.922$, chứng tỏ mô hình học tốt nhất khi toàn bộ thông tin pixel được giữ nguyên.
    \item \textbf{EDGE} xếp thứ hai với Accuracy $\approx 0.881$, thể hiện rằng thông tin về đường biên vẫn mang giá trị phân biệt mạnh.
    \item \textbf{BLOCK\_AVG} có Accuracy thấp nhất $\approx 0.870$, cho thấy việc giảm chiều mạnh làm mất nhiều chi tiết quan trọng.
\end{itemize}

=> Mô hình có feature đầy đủ (PIXEL) đạt performance tốt nhất, trong khi hai mô hình còn lại có xu hướng giảm độ chính xác khi thông tin đặc trưng bị giản lược.

\subsection{So sánh giữa các feature với nhau}

\subsubsection*{So sánh Accuracy}

\begin{figure}[H]
\centering
\includegraphics[width=0.9\linewidth]{img/chart_accuracy.png}
\caption{So sánh Accuracy giữa các feature}
\label{fig:chart_accuracy}
\end{figure}

Feature PIXEL đạt Accuracy cao nhất ($0.922$), tiếp theo là EDGE ($0.881$), và cuối cùng là BLOCK\_AVG ($0.8698$). 
=> Cho thấy mức độ chi tiết của feature ảnh hưởng trực tiếp đến khả năng phân loại của Softmax Regression.

\subsubsection*{So sánh Macro F1-score}

\begin{figure}[H]
\centering
\includegraphics[width=0.9\linewidth]{img/chart_macro_f1.png}
\caption{So sánh Macro F1-score giữa các feature}
\label{fig:chart_macro_f1}
\end{figure}

PIXEL tiếp tục đứng đầu với Macro F1 $\approx 0.9204$.  
EDGE đạt $\approx 0.8794$ và BLOCK\_AVG đạt khoảng $0.8672$.  
=> Thứ hạng các feature tương tự Accuracy, chứng tỏ hiệu năng ổn định trên toàn bộ lớp thay vì chỉ tập trung một vài lớp.

\subsubsection*{So sánh Precision và Recall}

\begin{figure}[H]
\centering
\includegraphics[width=0.9\linewidth]{img/chart_precision.png}
\caption{So sánh Macro Precision giữa các feature}
\label{fig:chart_precision}
\end{figure}

\begin{figure}[H]
\centering
\includegraphics[width=0.9\linewidth]{img/chart_recall.png}
\caption{So sánh Macro Recall giữa các feature}
\label{fig:chart_recall}
\end{figure}

Precision và Recall của PIXEL đều trên $0.92$, cao nhất trong ba feature. EDGE duy trì mức trung bình $\approx 0.88$ và BLOCK\_AVG thấp nhất khoảng $0.867$.  
=> Cả Precision và Recall đều cho thấy mô hình với feature BLOCK\_AVG có xu hướng bỏ sót và nhầm lẫn giữa các lớp nhiều hơn.

\subsection{Phân tích chi tiết qua Confusion Matrix}

Trong phần này, ta phân tích mức độ nhầm lẫn giữa các lớp dựa trên confusion matrix của từng feature.

\subsubsection*{PIXEL}

\begin{figure}[H]
\centering
\includegraphics[width=0.9\linewidth]{img/confusion_matrix_pixel_norm.png}
\caption{Confusion matrix chuẩn hoá – Feature PIXEL}
\label{fig:cm_pixel}
\end{figure}

Mô hình với PIXEL cho thấy đường chéo chính rất đậm:
\begin{itemize}
    \item Các lớp 0, 1, 6 đạt tỉ lệ dự đoán đúng trên 95\%.
    \item Model có 1 sự nhầm lẫn nhẹ giữa 3 và 5, phản ánh sự tương đồng hình dạng giữa 2 số này.
\end{itemize}

\subsubsection*{BLOCK\_AVG}

\begin{figure}[H]
\centering
\includegraphics[width=0.9\linewidth]{img/confusion_matrix_block_avg_norm.png}
\caption{Confusion matrix chuẩn hoá – Feature BLOCK\_AVG}
\label{fig:cm_block_avg}
\end{figure}

BLOCK\_AVG có mức nhầm lẫn cao hơn đáng kể:
\begin{itemize}
    \item Lớp 5 bị nhầm sang lớp 3 khoảng 6\%.
    \item Lớp 8 bị nhầm với 3, 5 và 9 (2–4\%).
\end{itemize}

=> Khi giảm ảnh xuống 7$\times$7, mô hình mất khả năng phân tách các lớp có cấu trúc tương tự.  
=> Tương tự như Lasso trong ví dụ mẫu, BLOCK\_AVG là feature “yếu nhất”.

\subsubsection*{EDGE}

\begin{figure}[H]
\centering
\includegraphics[width=0.9\linewidth]{img/confusion_matrix_edge_norm.png}
\caption{Confusion matrix chuẩn hoá – Feature EDGE}
\label{fig:cm_edge}
\end{figure}

EDGE thể hiện mức độ phân biệt khá tốt:
\begin{itemize}
    \item Các lớp có hình dạng rõ ràng như 0, 1, 7 được dự đoán chính xác cao.
    \item Tuy nhiên, các chữ số dựa nhiều vào chi tiết bên trong (5, 8, 9) có tỷ lệ nhầm lẫn lớn hơn.
\end{itemize}

=> EDGE tốt hơn BLOCK\_AVG nhưng chưa vượt được PIXEL vì thiếu thông tin cường độ pixel.

\subsection{Tổng kết và kết luận}

Từ toàn bộ số liệu và biểu đồ trên, ta có thể rút ra:

\begin{itemize}
    \item \textbf{PIXEL} là feature mạnh nhất, đạt toàn bộ metric cao nhất (Accuracy, Precision, Recall, F1-score).
    \item \textbf{EDGE} giữ lại cấu trúc biên đủ tốt nên đứng thứ hai.
    \item \textbf{BLOCK\_AVG} làm mất nhiều chi tiết quan trọng, dẫn đến performance thấp nhất.
\end{itemize}

=> Mặc dù BLOCK\_AVG và EDGE có thể giúp giảm số chiều và thời gian xử lý, chúng làm giảm đáng kể khả năng phân loại của Softmax Regression.

=> Tương tự cách Linear Regression với BIC vượt trội trong ví dụ mẫu, ở đây \textbf{PIXEL là lựa chọn tối ưu để huấn luyện Softmax Regression} trên bài toán nhận dạng chữ số.

