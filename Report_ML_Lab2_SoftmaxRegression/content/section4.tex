\pagebreak
\section{Đánh giá và phân tích kết quả}

\subsection{Kết quả dự đoán của mô hình với từng loại đặc trưng}

\begin{figure}[H]
\centering
\includegraphics[width=\linewidth]{img/comparison_all_metrics.png}
\caption{Biểu đồ so sánh các chỉ số Accuracy, Precision, Recall và F1-score giữa các phương pháp trích xuất đặc trưng}
\label{fig:compare_all_metrics}
\end{figure}
Kết quả thực nghiệm cho thấy cả ba phương pháp trích xuất đặc trưng (PIXEL, BLOCK\_AVG, EDGE) đều mang lại hiệu suất dự đoán khả quan, với độ chính xác dao động trong khoảng $0.87 - 0.92$. Tuy nhiên, có sự phân hóa rõ rệt về hiệu quả giữa các phương pháp:
\begin{itemize}
    \item \textbf{PIXEL} đạt độ chính xác (Accuracy) cao nhất $\approx 0.922$, cho thấy mô hình đạt hiệu suất tối ưu khi bảo toàn toàn bộ thông tin điểm ảnh.
    \item \textbf{EDGE} xếp thứ hai với Accuracy $\approx 0.881$, khẳng định rằng thông tin về đường biên vẫn mang giá trị phân loại lớn đối với tập dữ liệu này.
    \item \textbf{BLOCK\_AVG} có Accuracy thấp nhất $\approx 0.870$, phản ánh việc giảm chiều dữ liệu quá mức đã lược bỏ nhiều chi tiết thiết yếu cho quá trình phân lớp.
\end{itemize}

Như vậy, mô hình sử dụng đặc trưng nguyên bản (PIXEL) đạt hiệu năng tốt nhất, trong khi hai phương pháp còn lại cho thấy xu hướng giảm độ chính xác khi thông tin đặc trưng bị lược giản.

\subsection{So sánh giữa các đặc trưng với nhau}

\subsubsection*{So sánh về độ chính xác (Accuracy)}

\begin{figure}[H]
\centering
\includegraphics[width=0.9\linewidth]{img/chart_accuracy.png}
\caption{So sánh độ chính xác (Accuracy) giữa các đặc trưng}
\label{fig:chart_accuracy}
\end{figure}

Đặc trưng PIXEL dẫn đầu về độ chính xác ($0.922$), theo sau là EDGE ($0.881$) và cuối cùng là BLOCK\_AVG ($0.8698$). Điều này minh chứng rằng độ chi tiết của dữ liệu đầu vào có tác động trực tiếp đến khả năng phân loại của mô hình Softmax Regression.

\subsubsection*{So sánh về chỉ số Macro F1-score}

\begin{figure}[H]
\centering
\includegraphics[width=0.9\linewidth]{img/chart_macro_f1.png}
\caption{So sánh chỉ số Macro F1-score giữa các đặc trưng}
\label{fig:chart_macro_f1}
\end{figure}

PIXEL tiếp tục đứng đầu với Macro F1 $\approx 0.9204$. EDGE đạt $\approx 0.8794$ và BLOCK\_AVG đạt khoảng $0.8672$. Thứ hạng về F1-score tương đồng với Accuracy, cho thấy sự ổn định về hiệu năng của mô hình trên toàn bộ các lớp dữ liệu.

\subsubsection*{So sánh về Precision và Recall}

\begin{figure}[H]
\centering
\includegraphics[width=0.9\linewidth]{img/chart_precision.png}
\caption{So sánh Macro Precision giữa các đặc trưng}
\label{fig:chart_precision}
\end{figure}

\begin{figure}[H]
\centering
\includegraphics[width=0.9\linewidth]{img/chart_recall.png}
\caption{So sánh Macro Recall giữa các đặc trưng}
\label{fig:chart_recall}
\end{figure}

Cả Precision và Recall của PIXEL đều trên $0.92$, cao nhất trong ba đặc trưng. EDGE duy trì mức trung bình $\approx 0.88$ và BLOCK\_AVG thấp nhất khoảng $0.867$. Cả Precision và Recall đều cho thấy mô hình với feature BLOCK\_AVG có xu hướng bỏ sót và nhầm lẫn giữa các lớp nhiều hơn.

\subsection{Phân tích chi tiết qua ma trận nhầm lẫn (Confusion Matrix)}

Phần này tập trung phân tích mức độ phân loại sai giữa các lớp dựa trên Confusion Matrix đã được chuẩn hóa.

\subsubsection*{PIXEL}

\begin{figure}[H]
\centering
\includegraphics[width=0.9\linewidth]{img/confusion_matrix_pixel_norm.png}
\caption{Confusion matrix chuẩn hoá - Feature PIXEL}
\label{fig:cm_pixel}
\end{figure}

Đối với PIXEL, các giá trị trên đường chéo chính chiếm tỷ trọng lớn:
\begin{itemize}
    \item Các lớp 0, 1, 6 đạt tỉ lệ dự đoán đúng trên 95\%.
    \item Tồn tại một sự nhầm lẫn nhẹ giữa 3 và 5, phản ánh sự tương đồng hình dạng giữa hai số này.
\end{itemize}

\subsubsection*{BLOCK\_AVG}

\begin{figure}[H]
\centering
\includegraphics[width=0.9\linewidth]{img/confusion_matrix_block_avg_norm.png}
\caption{Confusion matrix chuẩn hoá - Feature BLOCK\_AVG}
\label{fig:cm_block_avg}
\end{figure}

BLOCK\_AVG có mức nhầm lẫn cao hơn đáng kể:
\begin{itemize}
    \item Lớp 5 bị nhầm sang lớp 3 khoảng 6\%.
    \item Lớp 8 bị nhầm với 1, 3, và 5 (2-4\%).
\end{itemize}
Cho thấy khi giảm ảnh xuống 7$\times$7, mô hình mắc nhiều sai sót hơn khi phân tách các lớp có cấu trúc tương tự.  

\subsubsection*{EDGE}

\begin{figure}[H]
\centering
\includegraphics[width=0.9\linewidth]{img/confusion_matrix_edge_norm.png}
\caption{Confusion matrix chuẩn hoá - Feature EDGE}
\label{fig:cm_edge}
\end{figure}

\begin{itemize}
    \item Các lớp có hình dạng rõ ràng như 0, 1, 6 được dự đoán chính xác cao.
    \item Tuy nhiên, tương tự như BLOCK\_AVG, EDGE cũng bị nhầm lẫn khi xác định các số còn lại, đặc biệt là 5 và 8.
\end{itemize}
Mặc dù EDGE vượt trội hơn BLOCK\_AVG, nó vẫn chưa đạt được hiệu suất của PIXEL do thiếu hụt thông tin về cường độ điểm ảnh. Tổng hợp các phân tích trên, ta có thể kết luận:

\begin{itemize}
    \item \textbf{PIXEL} là feature mạnh nhất, đạt toàn bộ metric cao nhất (Accuracy, Precision, Recall, F1-score).
    \item \textbf{EDGE} giữ lại cấu trúc biên đủ tốt nên đứng thứ hai.
    \item \textbf{BLOCK\_AVG} làm mất nhiều chi tiết quan trọng, dẫn đến performance thấp nhất.
\end{itemize}
Mặc dù BLOCK\_AVG và EDGE có ưu điểm về giảm số chiều dữ liệu và chi phí tính toán, việc áp dụng chúng đã dẫn đến sự suy giảm đáng kể khả năng phân loại của mô hình Softmax Regression trong bài toán này.